\secnumbersection{VALIDACIÓN DE LA SOLUCIÓN}

Se debe validar la solución propuesta. Esto significa probar o demostrar que la solución propuesta es válida para el entorno donde fue planteada.
Tradicionalmente es una etapa crítica, pues debe comprobarse por algún medio que vuestra propuesta es básicamente válida. En el caso de un desarrollo de software es la construcción y sus pruebas; en el caso de propuestas de modelos, guías o metodologías podrían ser desde la aplicación a un caso real hasta encuestas o entrevistas con especialistas; en el caso de mejoras de procesos u optimizaciones, podría ser comparar la situación actual (previa a la memoria) con la situación final (cuando la memoria está ya implementada) en base a un conjunto cuantitativo de indicadores o criterios.
