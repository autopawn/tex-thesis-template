\documentclass[letterpaper,12pt]{article}

% @@@@@@@@@@@@@@@@@@@@@@@@@@@@@@@@@@@@@@@@@@@@@@@@@@@@@@@@@@@@>
% VALORES A MODIFICAR POR USTED:
% @@@@@@@@@@@@@@@@@@@@@@@@@@@@@@@@@@@@@@@@@@@@@@@@@@@@@@@@@@@@>

% NOTE: Leer nota en el README sobre la font.

\newcommand{\titulo}{Título de la memoria}
\newcommand{\ciudad}{Ciudad} % e.g. Valparaíso
% TODO: Consultar el formato de los nombres:
\newcommand{\nombrealumno}{Nombre del alumno}
\newcommand{\nombreprofesor}{Nombre del profesor guía}
\newcommand{\nombrecorreferente}{Nombre del correferente}
% Mes y año del examen
\newcommand{\mesexamen}{Diciembre}
\newcommand{\anioexamen}{20XX}
% Dedicatoria y agradecimientos
\newcommand{\dedicatoria}{
Considerando lo importancia de este trabajo para los alumnos, este apartado es para que el autor entregue palabras personales para dedicar este documento. La extensión puede ser de máximo una hoja y se deben mantener este formato, tipo y tamaño de letra.
}
\newcommand{\agradecimientos}{
Considerando la importancia de este trabajo para los alumnos, este apartado se podrá incluir en el caso de que el autor desee agradecer a las personas que facilitaron alguna ayuda relevante en su trabajo para la realización de este documento. La extensión puede ser de máximo una hoja y se deben mantener este formato, tipo y tamaño de letra.
}
\newcommand{\resumen}{
El resumen y las palabras clave no deben superar la mitad de la página, donde debe precisarse brevemente: 1) lo que el autor ha hecho, 2) cómo lo hizo (sólo si es importante detallarlo), 3) los resultados principales, 4) la relevancia de los resultados. El resumen es una representación abreviada, pero comprensiva de la memoria y debe informar sobre el objetivo, la metodología y los resultados del trabajo realizado.
}
\newcommand{\resumeningles}{
Corresponde a la traducción al idioma inglés del Resumen anterior. Sujeto a la misma regla de extensión del Resumen.
}
\newcommand{\palabrasclave}{
Cinco es el máximo de palabras clave para describir los temas tratados en la memoria, ponerlas separadas por punto y comas.
}
\newcommand{\palabrasclaveingles}{
Corresponde a la traducción al idioma inglés de Palabras Clave anteriores.
}
% @@@@@@@@@@@@@@@@@@@@@@@@@@@@@@@@@@@@@@@@@@@@@@@@@@@@@@@@@@@@>

% Paquete para importar imágenes
\usepackage{graphicx}
% Directorio de las imágenes
\graphicspath{ {figures/} }

% Idioma y fuentes
\usepackage[spanish,es-tabla]{babel}
\usepackage[T1]{fontenc}

\usepackage{fontspec}
% Los siguientes comandos fueron sugeridos por @anibalbastiass (ver issue#5)
% para contar con Carlito en cursiva y negrita.
\setmainfont{Carlito}[BoldFont={* Bold}]
\setmainfont{Carlito}[ItalicFont={* Italic}]

% Paquete para definir cualquier tamaño de font
\usepackage{anyfontsize}

% Settear font
\setmainfont{Carlito}

% Tamaño de la página y márgenes
\usepackage[letterpaper,top=2.5cm,bottom=3cm,left=3cm,right=3cm,marginparwidth=1.75cm]{geometry}

% Determinar interlineado:
\renewcommand{\baselinestretch}{1.0}

% Eliminar sangrías:
\setlength{\parindent}{0cm}

% Paquete para definir los formatos de los títulos
\usepackage[explicit]{titlesec}

\titleformat{name=\section}[block]{\fontsize{16}{24}\selectfont\bfseries}{}{0pt}{#1}
\titleformat{name=\section,numberless}[block]{\fontsize{16}{24}\selectfont\bfseries}{}{0pt}{#1}
\titlespacing*{name=\section}{0pt}{0pt}{0.5cm}
\titlespacing*{name=\section,numberless}{0pt}{0pt}{0.5cm}

% Separación entre parrafos
\setlength{\parskip}{0.4cm}

% Paquetes de utilidad general
\usepackage{amsmath}
\usepackage{graphicx}
\usepackage{float}
\usepackage[colorlinks=true, allcolors=blue]{hyperref}

% Formato de las tablas de contenido
% \usepackage[tocflat]{tocstyle}
\usepackage{tocstyle}
\usetocstyle{allwithdot}

% Para obtener el número de la última página
\usepackage{lastpage}

% Header y footer
\usepackage{fancyhdr}
\fancypagestyle{portada}{
    \lhead{}
    \chead{}
    \rhead{}
    \lfoot{}
    \cfoot{\fontsize{10}{12}\selectfont \thepage}
    \rfoot{}
    \renewcommand{\headrulewidth}{0pt}
}
\fancypagestyle{intermedio}{
    \lhead{}
    \chead{\fontsize{10}{12}\selectfont\MakeUppercase{\titulo}}
    \rhead{}
    \lfoot{}
    \cfoot{\fontsize{10}{12}\selectfont Página \textbf{\thepage}\ de \textbf{\pageref{LastPage}}}
    \rfoot{}
    \renewcommand{\headrulewidth}{1pt}
}

% Comandos para secciones
\newcommand{\secnumbersection}[1]{
\addtocounter{section}{1}
\section*{CAPÍTULO \thesection \texorpdfstring{\\}\ #1}
\addcontentsline{toc}{section}{CAPÍTULO \thesection : #1}
\setcounter{subsection}{0}
}
\newcommand{\secnumberlesssection}[1]{
\section*{#1}
\phantomsection
\addcontentsline{toc}{section}{#1}
\setcounter{subsection}{0}
}

% Nombres
\addto\captionsspanish{\renewcommand{\contentsname}{ÍNDICE DE CONTENIDOS}}
\addto\captionsspanish{\renewcommand{\listfigurename}{ÍNDICE DE FIGURAS}}
\addto\captionsspanish{\renewcommand{\listtablename}{ÍNDICE DE TABLAS}}
\makeatletter
\renewenvironment{thebibliography}[1]
     {\secnumberlesssection{REFERENCIAS BIBLIOGRÁFICAS}
      \@mkboth{\MakeUppercase\bibname}{\MakeUppercase\bibname}%
      \list{\@biblabel{\@arabic\c@enumiv}}%
           {\settowidth\labelwidth{\@biblabel{#1}}%
            \leftmargin\labelwidth
            \advance\leftmargin\labelsep
            \@openbib@code
            \usecounter{enumiv}%
            \let\p@enumiv\@empty
            \renewcommand\theenumiv{\@arabic\c@enumiv}}%
      \sloppy
      \clubpenalty4000
      \@clubpenalty \clubpenalty
      \widowpenalty4000%
      \sfcode`\.\@m}
     {\def\@noitemerr
       {\@latex@warning{Empty `thebibliography' environment}}%
      \endlist}
\makeatother

% Personalizar Tabla de Contenidos

\usepackage{tocloft}
\renewcommand{\cftsecfont}{\fontsize{12}{14}\selectfont\fontspec{Carlito}}
\renewcommand{\cftsubsecfont}{\fontsize{12}{14}\selectfont\fontspec{Carlito}}
\renewcommand{\cftsubsubsecfont}{\fontsize{12}{14}\selectfont\fontspec{Carlito}}

\renewcommand\cftfigfont{\fontsize{12}{14}\selectfont\fontspec{Carlito}}

% Links sin color
\usepackage{hyperref}
\hypersetup{colorlinks = false}

% Comando para secciónes sin enumeración
% (sugerido por @anibalbastiass https://github.com/autopawn/tex-thesis-template/issues/5#issuecomment-916106128)
\newcommand{\secnumberlesssubsection}[1]{
\subsection*{#1}
\phantomsection
\addcontentsline{toc}{subsection}{#1}
\setcounter{subsection}{0}
}
% Forma de uso:
% \secnumberlesssubsection{"Sub seccion sin enumeración"}

% @@@@@@@@@@@@@@@@@@@@@@@@@@@@@@@@@@@@@@@@@@@@@@@@@@@@@@@@@@@@>
\begin{document}
\sloppy % Para evitar que referencias se escapen de los márgenes.

\pagestyle{portada}
\pagenumbering{roman}
\begin{titlepage}
\begin{center}
\noindent
{\fontsize{18}{22}\selectfont UNIVERSIDAD TÉCNICA FEDERICO SANTA MARÍA \\}
{\fontsize{16}{19}\selectfont DEPARTAMENTO DE INFORMÁTICA \\}
{\fontsize{16}{19}\selectfont \MakeUppercase{\ciudad}\ - CHILE \\}
\vspace{1.5cm}
\includegraphics[width=4.41cm,height=3.34cm]{logo/logo.jpg} \\
\vspace{1.5cm}
{\fontsize{20}{24}\selectfont ``\MakeUppercase{\titulo}'' \\}
\vfill
{\fontsize{16}{19}\selectfont \MakeUppercase{\nombrealumno} \\}
\vfill
{\fontsize{16}{19}\selectfont MEMORIA PARA OPTAR AL TÍTULO DE \\}
{\fontsize{16}{19}\selectfont INGENIERO CIVIL EN INFORMÁTICA \\}
\vspace{1.5cm}
{\fontsize{14}{17}\selectfont Profesor Guía: \nombreprofesor \\}
{\fontsize{14}{17}\selectfont Profesor Correferente: \nombrecorreferente \\}
\vspace{2.5cm}
{\fontsize{14}{17}\selectfont \mesexamen\ - \anioexamen \\}
\end{center}
\end{titlepage}

%@@@@@@@@@@@@@@@@@@@@@@@@@@@@@@@@@@@@@@@@@@@@@@@@@@@@@@@@@@@@@@
\
\vfill
\vfill
\begin{flushright}
\noindent {\fontsize{16}{19}\selectfont \textbf{DEDICATORIA} \\}
\end{flushright}
\begin{flushright}
\noindent \dedicatoria
\end{flushright}
\vfill
%@@@@@@@@@@@@@@@@@@@@@@@@@@@@@@@@@@@@@@@@@@@@@@@@@@@@@@@@@@@@@@
\newpage
\begin{center}
\noindent {\fontsize{16}{19}\selectfont \textbf{AGRADECIMIENTOS} \\}
\end{center}
%\vspace{0.25cm}
\agradecimientos
\vfill
%@@@@@@@@@@@@@@@@@@@@@@@@@@@@@@@@@@@@@@@@@@@@@@@@@@@@@@@@@@@@@@
\newpage
\section*{RESUMEN}
\vspace{0.5cm}
\noindent \textbf{Resumen---}\resumen \ \\
\ \\
\noindent \textbf{Palabras Clave---}\palabrasclave \ \\
% @@@@@
\noindent {\fontsize{16}{19}\selectfont \ \\ \ \\}
% @@@@@
\noindent {\fontsize{16}{19}\selectfont \textbf{ABSTRACT}}
\vspace{1.4cm}
\\
\noindent \textbf{\emph{Abstract}---}\resumeningles \ \\
\ \\
\noindent \textbf{\emph{Keywords}---}\palabrasclaveingles \ \\

%@@@@@@@@@@@@@@@@@@@@@@@@@@@@@@@@@@@@@@@@@@@@@@@@@@@@@@@@@@@@@@


\newpage
\section*{Glosario}

Aquí se deben colocar las siglas mencionadas en el trabajo y su explicación, por orden alfabético. Por ejemplo: \\

{\setlength{\parskip}{0cm} % Para evitar saltar entre cada elemento nombrado.
%Colocar aquí siglas:
DI: Departamento de Informática.

UTFSM: Universidad Técnica Federico Santa María.
}


%Índice de contenidos:
\newpage
\thispagestyle{portada}
\tableofcontents

%Índice de figuras:
\newpage
\thispagestyle{portada}
\phantomsection
\addcontentsline{toc}{section}{ÍNDICE DE FIGURAS}
\listoffigures
\phantomsection
\addcontentsline{toc}{section}{ÍNDICE DE TABLAS}
\listoftables

\newpage
\pagestyle{intermedio}
\pagenumbering{arabic}
\section*{Introducción}

Debe proporcionar a un lector los antecedentes suficientes para poder contextualizar en general la situación tratada, a través de una descripción breve del área de trabajo y del tema particular abordado, siendo bueno especificar la naturaleza y alcance del problema; así como describir el tipo de propuesta de solución que se realiza, esbozar la metodología a ser empleada e introducir a la estructura del documento mismo de la memoria.

En el fondo, que el lector al leer la Introducción pueda tener una síntesis de cómo fue desarrollada la memoria, a diferencia del Resumen dónde se explicita más qué se hizo, no cómo se hizo.


\newpage
\section{Capítulo \thesection \newline Definición del problema}

\newpage
\section{Capítulo \thesection \texorpdfstring{\\}\ Marco conceptual}

\newpage
\secnumbersection{PROPUESTA DE SOLUCIÓN}

Se debe desarrollar la solución propuesta. Los subcapítulos por poner aquí son propios del autor. Se sugiere mencionar metodología usada. Es conveniente incorporar figuras y tablas para aclarar la solución, que deben indicar el número de la figura, su nombre y su autor o fuente (si las diseñas tú, la fuente es ``Elaboración propia''). Ver ejemplos en esta página y en la siguiente.
Cabe mencionar que aquí está la esencia del trabajo en lo que se refiere al aporte creativo del memorista, es el momento de demostrar que usted es un destacado profesional que creó, diseñó y/o llevó a cabo la solución propuesta.

\newpage
\section{Capítulo \thesection \texorpdfstring{\\} Validación de la solución}

\newpage
\section{Capítulo \thesection \texorpdfstring{\\}\ Conclusiones}


\newpage
\secnumberlesssection{ANEXOS}

En los Anexos se incluye todo aquel material complementario que no es parte del contenido de los capítulos de la memoria, pero que permiten a un lector contar con un contenido adjunto relacionado con el tema.


\newpage
% Bibliografía estilo APA:
\bibliographystyle{apalike-es}
\bibliography{bibliografia}{}

\end{document}
